%!TEX program = xelatex

\documentclass[11pt,titlepage]{report}
%!TEX root = main.tex

\usepackage[T1]{fontenc}
\usepackage{lmodern}
\usepackage[svgnames]{xcolor}
\usepackage{fontspec} % XeLaTeX required!
\usepackage{graphicx}
\usepackage{circuitikz}
\usepackage{tikz}
\usepackage{pifont}
\usepackage[some]{background}
\usepackage{xltxtra} 
\usepackage{setspace}
\usepackage[absolute]{textpos}
\usepackage[latin1]{inputenc}
\usepackage[english]{babel}
\usepackage{graphicx}
\usepackage{wrapfig}
\usepackage{fullpage}
\usepackage[margin=1in]{geometry}
\usepackage{float}
\usepackage{url}
\usepackage{multicol}
\usepackage{hyperref}
\usepackage{titlepic}
\usepackage{standalone}
\usepackage{siunitx}
\usepackage{booktabs}
\usepackage{amsmath}
\usepackage{unicode-math}
\usepackage{verbatim}
\usepackage{enumitem}
\usepackage{listings}
\usepackage{multirow}
\usepackage{pgfplots}
\pgfplotsset{compat=1.8}
\usepackage{caption} 
\usepackage[parfill]{parskip}
\usepackage{import}
\usepackage[backend=bibtexu,texencoding=utf8,bibencoding=utf8,style=ieee,sortlocale=en_GB,language=auto]{biblatex}
\usepackage[strict,autostyle]{csquotes}
\usepackage[final]{pdfpages}
\usepackage{subcaption}
\usepackage{ifplatform}
%\captionsetup[table]{skip=10pt}


% Fix for includepdf bug in Mac OS X
\newcommand{\insertpdfpath}[1]{
	\ifwindows
	\newcommand{\insertpdf}[2]{\includepdf[pages=##1]{##2}}
	\else
	\newcommand{\insertpdf}[2]{\includepdf[pages=##1]{#1/##2}}
	\fi
}

%set fonts
\setmainfont[Ligatures=TeX]{Myriad Pro}
\setmathfont{Asana Math}
\setmonofont{Lucida Console}

\usepackage{titlesec, color}
\renewcommand{\familydefault}{\sfdefault} %set font family
\renewcommand{\arraystretch}{1.2} %set table vertical spacing
\setlength\parindent{0pt} %no paragraph indent
\hypersetup{ %setup hyperlinks
    colorlinks,
    citecolor=black,
    filecolor=black,
    linkcolor=black,
    urlcolor=black
}

%redesign chapter headings
\definecolor{gray75}{gray}{0.75}
\newcommand{\chapternumber}{\thechapter}
\newcommand{\hsp}{\hspace{20pt}}
\titleformat{\chapter}[hang]{\Huge\bfseries}{\chapternumber\hsp\textcolor{gray75}{|}\hsp}{0pt}{\Huge\bfseries}

%Redefine appendix headers
\renewcommand{\appendixname}{Appendix}
\renewcommand{\appendixtocname}{Appendices}
\renewcommand{\appendixpagename}{Appendices}

%For code listings
\definecolor{black}{rgb}{0,0,0}
\definecolor{browntags}{rgb}{0.65,0.1,0.1}
\definecolor{bluestrings}{rgb}{0,0,1}
\definecolor{graycomments}{rgb}{0.4,0.4,0.4}
\definecolor{redkeywords}{rgb}{1,0,0}
\definecolor{bluekeywords}{rgb}{0.13,0.13,0.8}
\definecolor{greencomments}{rgb}{0,0.5,0}
\definecolor{redstrings}{rgb}{0.9,0,0}
\definecolor{purpleidentifiers}{rgb}{0.01,0,0.01}


\lstdefinestyle{csharp}{
language=[Sharp]C,
showspaces=false,
showtabs=false,
breaklines=true,
showstringspaces=false,
breakatwhitespace=true,
escapeinside={(*@}{@*)},
columns=fullflexible,
commentstyle=\color{greencomments},
keywordstyle=\color{bluekeywords}\bfseries,
stringstyle=\color{redstrings},
identifierstyle=\color{purpleidentifiers},
basicstyle=\ttfamily\small}

\lstdefinestyle{c}{
language=C,
showspaces=false,
showtabs=false,
breaklines=true,
showstringspaces=false,
breakatwhitespace=true,
escapeinside={(*@}{@*)},
columns=fullflexible,
commentstyle=\color{greencomments},
keywordstyle=\color{bluekeywords}\bfseries,
stringstyle=\color{redstrings},
identifierstyle=\color{purpleidentifiers},
}

\lstdefinestyle{matlab}{
language=Matlab,
showspaces=false,
showtabs=false,
breaklines=true,
showstringspaces=false,
breakatwhitespace=true,
escapeinside={(*@}{@*)},
columns=fullflexible,
commentstyle=\color{greencomments},
keywordstyle=\color{bluekeywords}\bfseries,
stringstyle=\color{redstrings},
identifierstyle=\color{purpleidentifiers}
}

\lstdefinestyle{vhdl}{
language=VHDL,
showspaces=false,
showtabs=false,
breaklines=true,
showstringspaces=false,
breakatwhitespace=true,
escapeinside={(*@}{@*)},
columns=fullflexible,
commentstyle=\color{greencomments},
keywordstyle=\color{bluekeywords}\bfseries,
stringstyle=\color{redstrings},
identifierstyle=\color{purpleidentifiers}
}

\lstdefinestyle{xaml}{
language=XML,
showspaces=false,
showtabs=false,
breaklines=true,
showstringspaces=false,
breakatwhitespace=true,
escapeinside={(*@}{@*)},
columns=fullflexible,
commentstyle=\color{greencomments},
keywordstyle=\color{redkeywords},
stringstyle=\color{bluestrings},
tagstyle=\color{browntags},
morestring=[b]",
  morecomment=[s]{<?}{?>},
  morekeywords={xmlns,version,typex:AsyncRecords,x:Arguments,x:Boolean,x:Byte,x:Char,x:Class,x:ClassAttributes,x:ClassModifier,x:Code,x:ConnectionId,x:Decimal,x:Double,x:FactoryMethod,x:FieldModifier,x:Int16,x:Int32,x:Int64,x:Key,x:Members,x:Name,x:Object,x:Property,x:Shared,x:Single,x:String,x:Subclass,x:SynchronousMode,x:TimeSpan,x:TypeArguments,x:Uid,x:Uri,x:XData,Grid.Column,Grid.ColumnSpan,Click,ClipToBounds,Content,DropDownOpened,FontSize,Foreground,Header,Height,HorizontalAlignment,HorizontalContentAlignment,IsCancel,IsDefault,IsEnabled,IsSelected,Margin,MinHeight,MinWidth,Padding,SnapsToDevicePixels,Target,TextWrapping,Title,VerticalAlignment,VerticalContentAlignment,Width,WindowStartupLocation,Binding,Mode,OneWay,xmlns:x}
}

\lstdefinestyle{matlab}{
language=Matlab,
showspaces=false,
showtabs=false,
breaklines=true,
showstringspaces=false,
breakatwhitespace=true,
escapeinside={(*@}{@*)},
columns=fullflexible,
commentstyle=\color{greencomments},
keywordstyle=\color{bluekeywords}\bfseries,
stringstyle=\color{purpleidentifiers},
identifierstyle=\color{purpleidentifiers}
}

%defaults
\lstset{
basicstyle=\ttfamily\small,
extendedchars=false,
numbers=left,
numberstyle=\ttfamily\tiny,
stepnumber=1,
tabsize=4,
numbersep=5pt
}
\addbibresource{../../library/bibliography.bib}

\begin{document}

\chapter{Speed of sound}
\section{Derivation}
Let us derive an expression for the speed of sound with herein the dependency on environment temperature and humidity. From the Newton-Laplace equation for the speed of sound in a medium

\begin{equation}
	c = \sqrt{\frac{K}{\rho}},
\end{equation}

where $K$ is a coefficient of stiffness (the bulk modulus) and $\rho$ the density of the medium in conjunction with the fact that for a gas holds that

\begin{equation}
	K = \gamma p,
\end{equation}

where $\gamma$ denotes the \emph{adiabatic index} of the gas and $p$ its pressure, yields the following expression:

\begin{equation} \label{eq:c1}
	c = \sqrt{\gamma \frac{p}{\rho}}.
\end{equation}

Assuming $\gamma$ and $p$ (being the \emph{total air pressure}) are known, we now need to find an expression for the density of the water-air mixture, $\rho$.
To accomplish this, we utilize the ideal gas law

\begin{equation}
	p V = n R T,
\end{equation}

with $V$ the gas volume, $n$ the number of molecules, $R$ the specific gas constant and $T$ the absolute temperature in Kelvin, together with the definition of density $\rho = n / V$ to obtain:

\begin{equation}
	\rho = \frac{p}{R T}.
\end{equation}

Now, since the density of a mixture can be written as the sum of the densities of the individual gases, we can write:

\begin{equation} \label{eq:rho1}
	\rho_{air} = \frac{p_a}{R_a T} + \frac{p_w}{R_w T},
\end{equation}

where $p_a$, $R_a$, $p_w$ and $R_w$ are the pressures and specific gas constants for dry air and water vapor respectively. However, since we like to express air humidity as a ratio between the current partial water pressure $e_w$ and the saturation water pressure $e^*_w$ at a given temperature:

\begin{equation}
	\phi = \frac{e_w}{e^*_w} \times 100\%,
\end{equation}

we can use the fact that $p_a = p_{tot} - p_w$ and $p_w = e_w$ to rewrite \ref{eq:rho1}:

\begin{equation} \label{eq:rho2}
	\rho_{air} = \frac{R_w(100 p_{tot} - \phi e^*_w) + 100 R_a \phi e^*_w}{10^4 R_a R_w T},
\end{equation}

which we can then substitute back into \ref{eq:c1} to obtain the required expression:

\begin{equation} \label{eq:c2}
	c = \sqrt{\gamma \frac{10^4 p_{tot} R_a R_w T}{100 R_w(100 p_{tot} - \phi e^*_w) + 100 R_a \phi e^*_w}}.
\end{equation}
\cite{sengpiel-sound-speed,uiuc-rel-humid,wikipedia-speed-of-sound}

\section{Calculation}
Now, let us perform a calculation of the speed of sound in an environment of \SI{20}{\degree C} and a typical relative humidity of \SI{50}{\%}. We take $p_{tot}$ to be \SI{100}{kPa}, the standard atmospheric pressure, and it is given that $R_a = \SI{287.05}{J/kgK}$ and $R_w = \SI{461.495}{J/kgK}$ \cite{sengpiel-sound-speed}. Then we need to find a value for the adiabatic index $\gamma$. $\gamma$ is usually taken to be \num{1.67} for mono-atomic molecules, \num{1.4} for di-atomic molecules and \num{1.33} for tri-atomic molecules. Now, since air mostly consists of nitrogen and oxygen, which are di-atomic molecules, we take $\gamma = 1.4$ \cite{eng-tb-air-comp}. Lastly, we need a value for $e^*_w$. Buck \cite{buck-sat-press} presented an expression to approximate this value, but for this report it will suffice to make use of a table which presents the values of $e^*_w$ for different temperatures. In this manner, we find $e^*_w = \SI{2310}{Pa}$ for $T = \SI{20}{\degree C}$ \cite{eng-tb-sat-press}.
Substitution of the given values in \ref{eq:c2} yields:

\begin{equation}
	c = \sqrt{1.4 \frac{10^4 \cdot 100 \times 10^3 \cdot 287.05 \cdot 461.495 \cdot 293}{100 \cdot 461.495(100 \cdot 100 \times 10^3 - 50 \cdot 2310) + 100 \cdot 287.05 \cdot 50 \cdot 2310}} = \SI{343.90}{m/s}.
\end{equation}

\end{document}