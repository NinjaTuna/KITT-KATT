%!TEX program = xelatex 
\documentclass[11pt,titlepage]{report}
\usepackage[utf8]{inputenc}
\usepackage{amsmath}
\usepackage{amssymb}

\begin{document}

\newcommand{\mat}[1]{\mathbf{#1}}

\chapter*{Peak detection}
When we implemented localization, the question arised how long the microphones should record. Let us consider an recorded interval of length $I$. To determine the TDOAs, we must be able to fully record a peak. Let a peak be located at a position $X$, which is a random variable uniformly distributed as $X \sim U(0,I)$. The peaks are spaced a time $L$. By calculating the maximum TDOA, one can show that if $L/2 < X < I-L/2$, then this peak will almost surely result in a good localization. The localization rate is then given by
\begin{align*}
	R &= \frac{\operatorname{Prob}\left[\text{Correct localization}\right]}{\text{Interval length}} \\
	&= \frac{\int_{L/2}^{I-L/2}f_{X}(x)dx}{I} \\
	&= \frac{I-L}{I^2}.
\end{align*}
Optimizing the data rate yields
\begin{align*}
	\frac{dR}{dI} &= \frac{I^2-2I(I-L)}{I^4} = 0, \\
	I^2 &= 2IL, \\
	I &= 2L \hspace{10em} \text{($I$ cannot be zero)}.
\end{align*}
Therefore, we chose our recording time to be twice the peak interval length.
\end{document}