%!TEX program = xelatex 
\documentclass[11pt,titlepage]{report}
\usepackage[utf8]{inputenc}
\usepackage{amsmath}
\usepackage{amssymb}

\begin{document}

\newcommand{\mat}[1]{\mathbf{#1}}

\chapter*{Steering}
\section*{Analysis}
Let us consider a car with distance $L$ between both wheel axes. The rear axis of the car is located at $(0,0)$ and the front axis at $(L,0)$ in a way that they are parallel to each other. The front wheels are turned an angle $\phi$ relative to the positive $x$-axis. If the car drives at a speed $v$, then approximately, after a very small time $\Delta t$, the rear axis is located at $(v \cos{(\phi)}\Delta t,0)$ and the front axis at $(L+v \cos{(\phi)}\Delta t,v \sin{(\phi)}\Delta t)$. The car was first parallel to the positive $x$-axis. Therefore, if $\theta$ denotes the angle of the car relative to the positive $x$-axis,
\[
	\theta(t) = 0
\]
and
\[
	\theta(t+\Delta t)=\arctan{\left(\frac{v \sin{(\phi)}\Delta t}{L}\right)}.
\]
We can now evaluate $\dot{\theta}(t)$ by evaluating the limit
\begin{align}
	\dot{\theta}(t) &= \lim_{\Delta t \rightarrow 0} \frac{\theta(t+\Delta t)-\theta(t)}{\Delta t} \nonumber \\
	&= \lim_{\Delta t \rightarrow 0} \frac{\arctan{\left(\frac{v \sin{(\phi)}\Delta t}{L}\right)}}{\Delta t} \nonumber \\
	&= \lim_{\Delta t \rightarrow 0} \left(1+\frac{v^2 \sin{(\phi)}\Delta t^2}{L^2}\right)^{-1}\frac{v \sin{(\phi)}}{L} \quad \quad \text{(L'H\^opital's rule)} \nonumber \\
	&= \frac{v \sin{(\phi)}}{L}. \label{eq:steering-derivative}
\end{align}
Let us now consider a car whose trajectory is given by
\begin{align*}
	x(t) = R \cos{(\omega t)}, \\
	y(t) = R \sin{(\omega t)}.
\end{align*}
Here $R$ denotes the turning radius of the car. Note that
\[
	x^2(t)+y^2(t)=R^2.
\]
Differentiating both sides yields
\[
	\frac{dy}{dx}=-\frac{x(t)}{y(t)}.
\]
The angle of the car is then given by
\[
	\theta(t) = \arctan{(dy/dx)} = -\arctan{(1/\tan{\omega t})}.
\]
Note that
\[
	\dot{\theta}(t) = \frac{-1}{1+1/\tan^2{(\omega t)}} \frac{-1}{\tan^2{(\omega t)}} \frac{\omega}{\cos^2{(\omega t)}} = \omega.
\]
The angle of the car relative to the positive $x$-axis in this trajectory is given by
\[
	\alpha(t) = \omega t.
\]
Note that
\[
	\dot{\alpha}(t) = \omega = \dot{\theta}(t).
\]
Let $s$ denote the distance moved by the car and $v$ the speed of the car. Then, for some $\Delta t$ and $\Delta \alpha$,
\[
	s = v \Delta t = R \Delta \alpha.
\]
Now note that
\begin{align*}
	\frac{\Delta \alpha}{\Delta t} &= \frac{v}{R} \\
	&=\dot{\alpha}(t)=\dot{\theta}(t) = \frac{v \sin{(\phi)}}{L}.
\end{align*}
Therefore
\begin{equation} \label{eq:steering-radius}
	R = \frac{L}{\sin{(\phi)}}
\end{equation}

\section*{Control}
\subsection*{Feedback}
In the previous section we derived the relation
\[
	\dot{\theta}(\phi) = \frac{v \sin{(\phi)}}{L}.
\]
Let a feedback law be given by
\[
	\phi(\theta) = -k(\theta-\theta_{\text{ref}})
\]
where $k \in \mathbb{R}$, $k > 0$. Substituting yields the autonomous system
\[
	\dot{\theta}(\theta) = \frac{- v \sin{(k (\theta-\theta_{\text{ref}}))}}{L}.
\]
Solving $\dot{\theta}(\theta)=0$ yields our critical point of interest $\theta=\theta_{\text{ref}}$.

\subsection*{Stability}
Consider the Lyapunov function
\[
	V(\theta) = \frac{1}{2}(\theta-\theta_{\text{ref}})^2
\]
which is positive definite for $\left\{\theta | -\frac{\pi}{k} < \theta-\theta_{\text{ref}} < \frac{\pi}{k}, \theta \neq \theta_{\text{ref}} \right\}$. Now,
\begin{align*}
	\dot{V}(\theta) = (\theta-\theta_{\text{ref}}) \frac{- v \sin{(k ({\theta-\theta_{\text{ref}}}))}}{L}
\end{align*}
which is negative semi-definite for $\left\{\theta | -\frac{\pi}{k} < \theta-\theta_{\text{ref}} < \frac{\pi}{k}\right\}$. Therefore, we can conclude our critical point of interest $\theta=\theta_{\text{ref}}$ is asymptotically stable.

\subsection*{Linearisation}
Linearizing around $\theta=\theta_{\text{ref}}$ yields
\[
	\dot{\theta}(\theta) = \frac{- v k}{L} (\theta-\theta_{\text{ref}}).
\]
\end{document}