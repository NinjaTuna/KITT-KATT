%!TEX program = xelatex

\documentclass[11pt,titlepage]{report}
%!TEX root = main.tex

\usepackage[T1]{fontenc}
\usepackage{lmodern}
\usepackage[svgnames]{xcolor}
\usepackage{fontspec} % XeLaTeX required!
\usepackage{graphicx}
\usepackage{circuitikz}
\usepackage{tikz}
\usepackage{pifont}
\usepackage[some]{background}
\usepackage{xltxtra} 
\usepackage{setspace}
\usepackage[absolute]{textpos}
\usepackage[latin1]{inputenc}
\usepackage[english]{babel}
\usepackage{graphicx}
\usepackage{wrapfig}
\usepackage{fullpage}
\usepackage[margin=1in]{geometry}
\usepackage{float}
\usepackage{url}
\usepackage{multicol}
\usepackage{hyperref}
\usepackage{titlepic}
\usepackage{standalone}
\usepackage{siunitx}
\usepackage{booktabs}
\usepackage{amsmath}
\usepackage{unicode-math}
\usepackage{verbatim}
\usepackage{enumitem}
\usepackage{listings}
\usepackage{multirow}
\usepackage{pgfplots}
\pgfplotsset{compat=1.8}
\usepackage{caption} 
\usepackage[parfill]{parskip}
\usepackage{import}
\usepackage[backend=bibtexu,texencoding=utf8,bibencoding=utf8,style=ieee,sortlocale=en_GB,language=auto]{biblatex}
\usepackage[strict,autostyle]{csquotes}
\usepackage[final]{pdfpages}
\usepackage{subcaption}
\usepackage{ifplatform}
%\captionsetup[table]{skip=10pt}


% Fix for includepdf bug in Mac OS X
\newcommand{\insertpdfpath}[1]{
	\ifwindows
	\newcommand{\insertpdf}[2]{\includepdf[pages=##1]{##2}}
	\else
	\newcommand{\insertpdf}[2]{\includepdf[pages=##1]{#1/##2}}
	\fi
}

%set fonts
\setmainfont[Ligatures=TeX]{Myriad Pro}
\setmathfont{Asana Math}
\setmonofont{Lucida Console}

\usepackage{titlesec, color}
\renewcommand{\familydefault}{\sfdefault} %set font family
\renewcommand{\arraystretch}{1.2} %set table vertical spacing
\setlength\parindent{0pt} %no paragraph indent
\hypersetup{ %setup hyperlinks
    colorlinks,
    citecolor=black,
    filecolor=black,
    linkcolor=black,
    urlcolor=black
}

%redesign chapter headings
\definecolor{gray75}{gray}{0.75}
\newcommand{\chapternumber}{\thechapter}
\newcommand{\hsp}{\hspace{20pt}}
\titleformat{\chapter}[hang]{\Huge\bfseries}{\chapternumber\hsp\textcolor{gray75}{|}\hsp}{0pt}{\Huge\bfseries}

%Redefine appendix headers
\renewcommand{\appendixname}{Appendix}
\renewcommand{\appendixtocname}{Appendices}
\renewcommand{\appendixpagename}{Appendices}

%For code listings
\definecolor{black}{rgb}{0,0,0}
\definecolor{browntags}{rgb}{0.65,0.1,0.1}
\definecolor{bluestrings}{rgb}{0,0,1}
\definecolor{graycomments}{rgb}{0.4,0.4,0.4}
\definecolor{redkeywords}{rgb}{1,0,0}
\definecolor{bluekeywords}{rgb}{0.13,0.13,0.8}
\definecolor{greencomments}{rgb}{0,0.5,0}
\definecolor{redstrings}{rgb}{0.9,0,0}
\definecolor{purpleidentifiers}{rgb}{0.01,0,0.01}


\lstdefinestyle{csharp}{
language=[Sharp]C,
showspaces=false,
showtabs=false,
breaklines=true,
showstringspaces=false,
breakatwhitespace=true,
escapeinside={(*@}{@*)},
columns=fullflexible,
commentstyle=\color{greencomments},
keywordstyle=\color{bluekeywords}\bfseries,
stringstyle=\color{redstrings},
identifierstyle=\color{purpleidentifiers},
basicstyle=\ttfamily\small}

\lstdefinestyle{c}{
language=C,
showspaces=false,
showtabs=false,
breaklines=true,
showstringspaces=false,
breakatwhitespace=true,
escapeinside={(*@}{@*)},
columns=fullflexible,
commentstyle=\color{greencomments},
keywordstyle=\color{bluekeywords}\bfseries,
stringstyle=\color{redstrings},
identifierstyle=\color{purpleidentifiers},
}

\lstdefinestyle{matlab}{
language=Matlab,
showspaces=false,
showtabs=false,
breaklines=true,
showstringspaces=false,
breakatwhitespace=true,
escapeinside={(*@}{@*)},
columns=fullflexible,
commentstyle=\color{greencomments},
keywordstyle=\color{bluekeywords}\bfseries,
stringstyle=\color{redstrings},
identifierstyle=\color{purpleidentifiers}
}

\lstdefinestyle{vhdl}{
language=VHDL,
showspaces=false,
showtabs=false,
breaklines=true,
showstringspaces=false,
breakatwhitespace=true,
escapeinside={(*@}{@*)},
columns=fullflexible,
commentstyle=\color{greencomments},
keywordstyle=\color{bluekeywords}\bfseries,
stringstyle=\color{redstrings},
identifierstyle=\color{purpleidentifiers}
}

\lstdefinestyle{xaml}{
language=XML,
showspaces=false,
showtabs=false,
breaklines=true,
showstringspaces=false,
breakatwhitespace=true,
escapeinside={(*@}{@*)},
columns=fullflexible,
commentstyle=\color{greencomments},
keywordstyle=\color{redkeywords},
stringstyle=\color{bluestrings},
tagstyle=\color{browntags},
morestring=[b]",
  morecomment=[s]{<?}{?>},
  morekeywords={xmlns,version,typex:AsyncRecords,x:Arguments,x:Boolean,x:Byte,x:Char,x:Class,x:ClassAttributes,x:ClassModifier,x:Code,x:ConnectionId,x:Decimal,x:Double,x:FactoryMethod,x:FieldModifier,x:Int16,x:Int32,x:Int64,x:Key,x:Members,x:Name,x:Object,x:Property,x:Shared,x:Single,x:String,x:Subclass,x:SynchronousMode,x:TimeSpan,x:TypeArguments,x:Uid,x:Uri,x:XData,Grid.Column,Grid.ColumnSpan,Click,ClipToBounds,Content,DropDownOpened,FontSize,Foreground,Header,Height,HorizontalAlignment,HorizontalContentAlignment,IsCancel,IsDefault,IsEnabled,IsSelected,Margin,MinHeight,MinWidth,Padding,SnapsToDevicePixels,Target,TextWrapping,Title,VerticalAlignment,VerticalContentAlignment,Width,WindowStartupLocation,Binding,Mode,OneWay,xmlns:x}
}

\lstdefinestyle{matlab}{
language=Matlab,
showspaces=false,
showtabs=false,
breaklines=true,
showstringspaces=false,
breakatwhitespace=true,
escapeinside={(*@}{@*)},
columns=fullflexible,
commentstyle=\color{greencomments},
keywordstyle=\color{bluekeywords}\bfseries,
stringstyle=\color{purpleidentifiers},
identifierstyle=\color{purpleidentifiers}
}

%defaults
\lstset{
basicstyle=\ttfamily\small,
extendedchars=false,
numbers=left,
numberstyle=\ttfamily\tiny,
stepnumber=1,
tabsize=4,
numbersep=5pt
}
\addbibresource{../../library/bibliography.bib}

\begin{document}

\chapter{Assignment 2}
\section{Controlling the output}
Consider the state-space model

\begin{align}
	\dot{\vec{x}} &= \mat{A}\vec{x} + \mat{B}\vec{u}, \\
	\vec{y} &= \mat{C} \vec{x}. \label{eq:ass-2-model-output}
\end{align}

Let

\begin{equation}
	\vec{u} = -\mat{K}\vec{x} + \vec{r}
\end{equation}

be a feedback law which renders the considered state-space system asymptotically stable. Substituting $\vec{u}$ yields

\begin{equation} \label{eq:ass-2-derivative}
	\dot{\vec{x}} = (\mat{A} - \mat{B} \mat{K}) \vec{x} + \mat{B} \vec{r}.
\end{equation}

Using the fact that our system is asymptotically stable, we can argue that $\dot{\vec{x}} \to 0$ when $t \to \infty$. Combining Equation~\ref{eq:ass-2-model-output} and \ref{eq:ass-2-derivative} yields

\begin{equation}
	\vec{y} \to \mat{C} (\mat{B} \mat{K} - \mat{A})^{-1} \mat{B} \vec{r} \text{ when } t \to \infty.
\end{equation}

Therefore, the output $\vec{y}$ converges to the scaled applied input $\vec{r}$. We can utilize this fact to control the output of the model.

\section{The controller}
Putting together the compensator, observer and output reference leaves us at the controller given by

\begin{align}
	\dot{\hat{\vec{x}}} &= (\mat{A}-\mat{L}\mat{C}-\mat{B}\mat{K}) \hat{\vec{x}} + \mat{B}(\mat{C}(\mat{B} \mat{K} - \mat{A})^{-1} \mat{B})^{-1} \vec{r} + \mat{L} \vec{y}, \\
	\vec{x} &= (\mat{C}(\mat{B} \mat{K} - \mat{A})^{-1} \mat{B})^{-1} \vec{r} - \mat{K} \hat{\vec{x}}.
\end{align}

Here the drive excitation is denoted by $\vec{x}$ and distance measured by $\vec{y}$. However, we can only regulate the throttle of KITT approximately once per \SI{300}{ms}. We must therefore discretize the controller. Discretization using Heun's method \cite{wikipedia-heuns} yields

\begin{align}
	f(\vec{x},\vec{r},\vec{y}) &= (\mat{A}-\mat{L}\mat{C}-\mat{B}\mat{K}) \hat{\vec{x}} + \mat{B}(\mat{C}(\mat{B} \mat{K} - \mat{A})^{-1} \mat{B})^{-1} \vec{r} + \mat{L} \vec{y}, \\
	\hat{\vec{x}}_n &= \hat{\vec{x}}_{n-1} + \frac{T}{2} \left( f\left(\vec{x}_{n-1},\vec{r}_{n-1},\vec{y}_{n-1} \right) + f\left( \vec{x}_{n-1} + T f(\vec{x}_{n-1},\vec{r}_{n-1},\vec{y}_{n-1}),\vec{r}_{n},\vec{y}_{n}\right) \right), \\
	\vec{x}_n &= (\mat{C}(\mat{B} \mat{K} - \mat{A})^{-1} \mat{B})^{-1} \vec{r}_n - \mat{K} \hat{\vec{x}}_n.
\end{align}

Heun's method averages the slopes at points $n$ and $n+1$ using Euler's method to predict the value at point $n+1$. This gives us a decent and relatively easy to implement approximation.


\end{document}