%!TEX program = xelatex

\documentclass[11pt,titlepage]{report}
%!TEX root = main.tex

\usepackage[T1]{fontenc}
\usepackage{lmodern}
\usepackage[svgnames]{xcolor}
\usepackage{fontspec} % XeLaTeX required!
\usepackage{graphicx}
\usepackage{circuitikz}
\usepackage{tikz}
\usepackage{pifont}
\usepackage[some]{background}
\usepackage{xltxtra} 
\usepackage{setspace}
\usepackage[absolute]{textpos}
\usepackage[latin1]{inputenc}
\usepackage[english]{babel}
\usepackage{graphicx}
\usepackage{wrapfig}
\usepackage{fullpage}
\usepackage[margin=1in]{geometry}
\usepackage{float}
\usepackage{url}
\usepackage{multicol}
\usepackage{hyperref}
\usepackage{titlepic}
\usepackage{standalone}
\usepackage{siunitx}
\usepackage{booktabs}
\usepackage{amsmath}
\usepackage{unicode-math}
\usepackage{verbatim}
\usepackage{enumitem}
\usepackage{listings}
\usepackage{multirow}
\usepackage{pgfplots}
\pgfplotsset{compat=1.8}
\usepackage{caption} 
\usepackage[parfill]{parskip}
\usepackage{import}
\usepackage[backend=bibtexu,texencoding=utf8,bibencoding=utf8,style=ieee,sortlocale=en_GB,language=auto]{biblatex}
\usepackage[strict,autostyle]{csquotes}
\usepackage[final]{pdfpages}
\usepackage{subcaption}
\usepackage{ifplatform}
%\captionsetup[table]{skip=10pt}


% Fix for includepdf bug in Mac OS X
\newcommand{\insertpdfpath}[1]{
	\ifwindows
	\newcommand{\insertpdf}[2]{\includepdf[pages=##1]{##2}}
	\else
	\newcommand{\insertpdf}[2]{\includepdf[pages=##1]{#1/##2}}
	\fi
}

%set fonts
\setmainfont[Ligatures=TeX]{Myriad Pro}
\setmathfont{Asana Math}
\setmonofont{Lucida Console}

\usepackage{titlesec, color}
\renewcommand{\familydefault}{\sfdefault} %set font family
\renewcommand{\arraystretch}{1.2} %set table vertical spacing
\setlength\parindent{0pt} %no paragraph indent
\hypersetup{ %setup hyperlinks
    colorlinks,
    citecolor=black,
    filecolor=black,
    linkcolor=black,
    urlcolor=black
}

%redesign chapter headings
\definecolor{gray75}{gray}{0.75}
\newcommand{\chapternumber}{\thechapter}
\newcommand{\hsp}{\hspace{20pt}}
\titleformat{\chapter}[hang]{\Huge\bfseries}{\chapternumber\hsp\textcolor{gray75}{|}\hsp}{0pt}{\Huge\bfseries}

%Redefine appendix headers
\renewcommand{\appendixname}{Appendix}
\renewcommand{\appendixtocname}{Appendices}
\renewcommand{\appendixpagename}{Appendices}

%For code listings
\definecolor{black}{rgb}{0,0,0}
\definecolor{browntags}{rgb}{0.65,0.1,0.1}
\definecolor{bluestrings}{rgb}{0,0,1}
\definecolor{graycomments}{rgb}{0.4,0.4,0.4}
\definecolor{redkeywords}{rgb}{1,0,0}
\definecolor{bluekeywords}{rgb}{0.13,0.13,0.8}
\definecolor{greencomments}{rgb}{0,0.5,0}
\definecolor{redstrings}{rgb}{0.9,0,0}
\definecolor{purpleidentifiers}{rgb}{0.01,0,0.01}


\lstdefinestyle{csharp}{
language=[Sharp]C,
showspaces=false,
showtabs=false,
breaklines=true,
showstringspaces=false,
breakatwhitespace=true,
escapeinside={(*@}{@*)},
columns=fullflexible,
commentstyle=\color{greencomments},
keywordstyle=\color{bluekeywords}\bfseries,
stringstyle=\color{redstrings},
identifierstyle=\color{purpleidentifiers},
basicstyle=\ttfamily\small}

\lstdefinestyle{c}{
language=C,
showspaces=false,
showtabs=false,
breaklines=true,
showstringspaces=false,
breakatwhitespace=true,
escapeinside={(*@}{@*)},
columns=fullflexible,
commentstyle=\color{greencomments},
keywordstyle=\color{bluekeywords}\bfseries,
stringstyle=\color{redstrings},
identifierstyle=\color{purpleidentifiers},
}

\lstdefinestyle{matlab}{
language=Matlab,
showspaces=false,
showtabs=false,
breaklines=true,
showstringspaces=false,
breakatwhitespace=true,
escapeinside={(*@}{@*)},
columns=fullflexible,
commentstyle=\color{greencomments},
keywordstyle=\color{bluekeywords}\bfseries,
stringstyle=\color{redstrings},
identifierstyle=\color{purpleidentifiers}
}

\lstdefinestyle{vhdl}{
language=VHDL,
showspaces=false,
showtabs=false,
breaklines=true,
showstringspaces=false,
breakatwhitespace=true,
escapeinside={(*@}{@*)},
columns=fullflexible,
commentstyle=\color{greencomments},
keywordstyle=\color{bluekeywords}\bfseries,
stringstyle=\color{redstrings},
identifierstyle=\color{purpleidentifiers}
}

\lstdefinestyle{xaml}{
language=XML,
showspaces=false,
showtabs=false,
breaklines=true,
showstringspaces=false,
breakatwhitespace=true,
escapeinside={(*@}{@*)},
columns=fullflexible,
commentstyle=\color{greencomments},
keywordstyle=\color{redkeywords},
stringstyle=\color{bluestrings},
tagstyle=\color{browntags},
morestring=[b]",
  morecomment=[s]{<?}{?>},
  morekeywords={xmlns,version,typex:AsyncRecords,x:Arguments,x:Boolean,x:Byte,x:Char,x:Class,x:ClassAttributes,x:ClassModifier,x:Code,x:ConnectionId,x:Decimal,x:Double,x:FactoryMethod,x:FieldModifier,x:Int16,x:Int32,x:Int64,x:Key,x:Members,x:Name,x:Object,x:Property,x:Shared,x:Single,x:String,x:Subclass,x:SynchronousMode,x:TimeSpan,x:TypeArguments,x:Uid,x:Uri,x:XData,Grid.Column,Grid.ColumnSpan,Click,ClipToBounds,Content,DropDownOpened,FontSize,Foreground,Header,Height,HorizontalAlignment,HorizontalContentAlignment,IsCancel,IsDefault,IsEnabled,IsSelected,Margin,MinHeight,MinWidth,Padding,SnapsToDevicePixels,Target,TextWrapping,Title,VerticalAlignment,VerticalContentAlignment,Width,WindowStartupLocation,Binding,Mode,OneWay,xmlns:x}
}

\lstdefinestyle{matlab}{
language=Matlab,
showspaces=false,
showtabs=false,
breaklines=true,
showstringspaces=false,
breakatwhitespace=true,
escapeinside={(*@}{@*)},
columns=fullflexible,
commentstyle=\color{greencomments},
keywordstyle=\color{bluekeywords}\bfseries,
stringstyle=\color{purpleidentifiers},
identifierstyle=\color{purpleidentifiers}
}

%defaults
\lstset{
basicstyle=\ttfamily\small,
extendedchars=false,
numbers=left,
numberstyle=\ttfamily\tiny,
stepnumber=1,
tabsize=4,
numbersep=5pt
}
\addbibresource{../../library/bibliography.bib}

\begin{document}

\section{Analysis of the forces on a car}
From the motor's perspective, the moment of intertia seen is

\begin{equation}
	J_{eq} = J_m + \frac{1}{N^2} J_L
\end{equation}

in which $N$ represents the gear ratio, $J_m$ the moment of intertia of the motor and $J_L$ the moment of intertia of the load. If an external force, such as air resistance, is applied to the car, then the torque which the motor has to deliver is given by

\begin{equation} \label{eq:ass-4-eq-torque}
	\tau_{m} = \alpha_m (J_m + \frac{1}{N^2} J_L) + \frac{1}{N} \tau_{ex} =
	\frac{a_v N}{r_{wheel}} (J_m + \frac{1}{N^2} J_{wheels}) + r_{wheel} \frac{F_{ex}}{N}.
\end{equation}

Here the acceleration of the moment is denoted by $\alpha_m$, the external torque by $\tau_{ex}$, the acceleration of the vehicle by $a_v$, the radius of the wheels by $r_{wheel}$, the moment of inertia of the wheels by $J_{wheels}$ and the external force by $F_{ex}$. Letting the external force be the aerodynamic drag, gravitational force and rolling resistance yields

\begin{equation}
	\tau_{m} =
	\frac{a_v N}{r_{wheel}} (J_m + \frac{1}{N^2} J_{wheels}) + r_{wheel} \frac{
		m_v g \sin{(\theta)} +
		m_v g \cos{(\theta)} C_r +
		\frac{1}{2} C_w \rho A_f v_{rel}^2
	}{N}.
\end{equation}

Here the rolling resistance coefficient is denoted by $C_r$, the aerodynamic drag coefficient by $C_w$, the mass of the vehicle and passengers by $m_v$, the frontal area by $A_f$, the vehicle's speed relative to the wind $v_{rel}$ and the angle of the inclination of the road by $\theta$. If the power transmission's efficiency is $\eta$ and the vehicle is moving at a speed $v$, then the total power the motor has to deliver is

\begin{equation} \label{eq:ass-3-driving}
	P_{m} = \frac{\omega_m \tau_{m}}{\eta} =
	\frac{N v}{r_{wheel} \eta} \left(
		\frac{a_v N}{r_{wheel}} (J_m + \frac{1}{N^2} J_{wheels}) + r_{wheel} \frac{
			m_v g \sin{(\theta)} +
			m_v g \cos{(\theta)} C_r +
			\frac{1}{2} C_w \rho A_f v_{rel}^2
		}{N}
	\right).
\end{equation}

However, this equation is invalid if the car breaks regeneratively. In this case, the external torque drives the motor. If the efficiency of the power transmission is $\eta_r$, then the power delivered to the motor is given by

\begin{equation} \label{eq:ass-3-regen}
	P_{mr} = \eta_r \omega_m \tau_{m} =
	\frac{N v \eta_r}{r_{wheel}} \left(
		\frac{a_v N}{r_{wheel}} (J_m + \frac{1}{N^2} J_{wheels}) + r_{wheel} \frac{
			m_v g \sin{(\theta)} +
			m_v g \cos{(\theta)} C_r +
			\frac{1}{2} C_w \rho A_f v_{rel}^2
		}{N}
	\right).
\end{equation}

Here the power delivered to the motor is denoted by $P_{mr}$. One should keep in mind that the acceleration of the vehicle should be negative is the vehicle is breaking. Using Equations~\ref{eq:ass-3-driving} and \ref{eq:ass-3-regen}, we were able to solve the questions asked in Exercise 3. We assumed that the rolling resistance coefficient was approximately \num{0.01}, the aerodynamic drag coefficient approximately \num{0.25}, the speed of the wind was \SI{0}{m/s} and there was no inclination of the road. The power needed to for the car to drive at \SI{10}{m/s} and accelerating at \SI{5}{m/s^2} is approximately \SI{4.6}{kW}. The power gained from regenerative breaking with an acceleration of \SI{-2}{m/s} at \SI{30}{m/s} is approximately \SI{6.3}{kW}.

\end{document}