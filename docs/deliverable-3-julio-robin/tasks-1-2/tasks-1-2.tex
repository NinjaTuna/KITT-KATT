%!TEX program = xelatex

\documentclass[11pt,titlepage]{report}
%!TEX root = main.tex

\usepackage[T1]{fontenc}
\usepackage{lmodern}
\usepackage[svgnames]{xcolor}
\usepackage{fontspec} % XeLaTeX required!
\usepackage{graphicx}
\usepackage{circuitikz}
\usepackage{tikz}
\usepackage{pifont}
\usepackage[some]{background}
\usepackage{xltxtra} 
\usepackage{setspace}
\usepackage[absolute]{textpos}
\usepackage[latin1]{inputenc}
\usepackage[english]{babel}
\usepackage{graphicx}
\usepackage{wrapfig}
\usepackage{fullpage}
\usepackage[margin=1in]{geometry}
\usepackage{float}
\usepackage{url}
\usepackage{multicol}
\usepackage{hyperref}
\usepackage{titlepic}
\usepackage{standalone}
\usepackage{siunitx}
\usepackage{booktabs}
\usepackage{amsmath}
\usepackage{unicode-math}
\usepackage{verbatim}
\usepackage{enumitem}
\usepackage{listings}
\usepackage{multirow}
\usepackage{pgfplots}
\pgfplotsset{compat=1.8}
\usepackage{caption} 
\usepackage[parfill]{parskip}
\usepackage{import}
\usepackage[backend=bibtexu,texencoding=utf8,bibencoding=utf8,style=ieee,sortlocale=en_GB,language=auto]{biblatex}
\usepackage[strict,autostyle]{csquotes}
\usepackage[final]{pdfpages}
\usepackage{subcaption}
\usepackage{ifplatform}
%\captionsetup[table]{skip=10pt}


% Fix for includepdf bug in Mac OS X
\newcommand{\insertpdfpath}[1]{
	\ifwindows
	\newcommand{\insertpdf}[2]{\includepdf[pages=##1]{##2}}
	\else
	\newcommand{\insertpdf}[2]{\includepdf[pages=##1]{#1/##2}}
	\fi
}

%set fonts
\setmainfont[Ligatures=TeX]{Myriad Pro}
\setmathfont{Asana Math}
\setmonofont{Lucida Console}

\usepackage{titlesec, color}
\renewcommand{\familydefault}{\sfdefault} %set font family
\renewcommand{\arraystretch}{1.2} %set table vertical spacing
\setlength\parindent{0pt} %no paragraph indent
\hypersetup{ %setup hyperlinks
    colorlinks,
    citecolor=black,
    filecolor=black,
    linkcolor=black,
    urlcolor=black
}

%redesign chapter headings
\definecolor{gray75}{gray}{0.75}
\newcommand{\chapternumber}{\thechapter}
\newcommand{\hsp}{\hspace{20pt}}
\titleformat{\chapter}[hang]{\Huge\bfseries}{\chapternumber\hsp\textcolor{gray75}{|}\hsp}{0pt}{\Huge\bfseries}

%Redefine appendix headers
\renewcommand{\appendixname}{Appendix}
\renewcommand{\appendixtocname}{Appendices}
\renewcommand{\appendixpagename}{Appendices}

%For code listings
\definecolor{black}{rgb}{0,0,0}
\definecolor{browntags}{rgb}{0.65,0.1,0.1}
\definecolor{bluestrings}{rgb}{0,0,1}
\definecolor{graycomments}{rgb}{0.4,0.4,0.4}
\definecolor{redkeywords}{rgb}{1,0,0}
\definecolor{bluekeywords}{rgb}{0.13,0.13,0.8}
\definecolor{greencomments}{rgb}{0,0.5,0}
\definecolor{redstrings}{rgb}{0.9,0,0}
\definecolor{purpleidentifiers}{rgb}{0.01,0,0.01}


\lstdefinestyle{csharp}{
language=[Sharp]C,
showspaces=false,
showtabs=false,
breaklines=true,
showstringspaces=false,
breakatwhitespace=true,
escapeinside={(*@}{@*)},
columns=fullflexible,
commentstyle=\color{greencomments},
keywordstyle=\color{bluekeywords}\bfseries,
stringstyle=\color{redstrings},
identifierstyle=\color{purpleidentifiers},
basicstyle=\ttfamily\small}

\lstdefinestyle{c}{
language=C,
showspaces=false,
showtabs=false,
breaklines=true,
showstringspaces=false,
breakatwhitespace=true,
escapeinside={(*@}{@*)},
columns=fullflexible,
commentstyle=\color{greencomments},
keywordstyle=\color{bluekeywords}\bfseries,
stringstyle=\color{redstrings},
identifierstyle=\color{purpleidentifiers},
}

\lstdefinestyle{matlab}{
language=Matlab,
showspaces=false,
showtabs=false,
breaklines=true,
showstringspaces=false,
breakatwhitespace=true,
escapeinside={(*@}{@*)},
columns=fullflexible,
commentstyle=\color{greencomments},
keywordstyle=\color{bluekeywords}\bfseries,
stringstyle=\color{redstrings},
identifierstyle=\color{purpleidentifiers}
}

\lstdefinestyle{vhdl}{
language=VHDL,
showspaces=false,
showtabs=false,
breaklines=true,
showstringspaces=false,
breakatwhitespace=true,
escapeinside={(*@}{@*)},
columns=fullflexible,
commentstyle=\color{greencomments},
keywordstyle=\color{bluekeywords}\bfseries,
stringstyle=\color{redstrings},
identifierstyle=\color{purpleidentifiers}
}

\lstdefinestyle{xaml}{
language=XML,
showspaces=false,
showtabs=false,
breaklines=true,
showstringspaces=false,
breakatwhitespace=true,
escapeinside={(*@}{@*)},
columns=fullflexible,
commentstyle=\color{greencomments},
keywordstyle=\color{redkeywords},
stringstyle=\color{bluestrings},
tagstyle=\color{browntags},
morestring=[b]",
  morecomment=[s]{<?}{?>},
  morekeywords={xmlns,version,typex:AsyncRecords,x:Arguments,x:Boolean,x:Byte,x:Char,x:Class,x:ClassAttributes,x:ClassModifier,x:Code,x:ConnectionId,x:Decimal,x:Double,x:FactoryMethod,x:FieldModifier,x:Int16,x:Int32,x:Int64,x:Key,x:Members,x:Name,x:Object,x:Property,x:Shared,x:Single,x:String,x:Subclass,x:SynchronousMode,x:TimeSpan,x:TypeArguments,x:Uid,x:Uri,x:XData,Grid.Column,Grid.ColumnSpan,Click,ClipToBounds,Content,DropDownOpened,FontSize,Foreground,Header,Height,HorizontalAlignment,HorizontalContentAlignment,IsCancel,IsDefault,IsEnabled,IsSelected,Margin,MinHeight,MinWidth,Padding,SnapsToDevicePixels,Target,TextWrapping,Title,VerticalAlignment,VerticalContentAlignment,Width,WindowStartupLocation,Binding,Mode,OneWay,xmlns:x}
}

\lstdefinestyle{matlab}{
language=Matlab,
showspaces=false,
showtabs=false,
breaklines=true,
showstringspaces=false,
breakatwhitespace=true,
escapeinside={(*@}{@*)},
columns=fullflexible,
commentstyle=\color{greencomments},
keywordstyle=\color{bluekeywords}\bfseries,
stringstyle=\color{purpleidentifiers},
identifierstyle=\color{purpleidentifiers}
}

%defaults
\lstset{
basicstyle=\ttfamily\small,
extendedchars=false,
numbers=left,
numberstyle=\ttfamily\tiny,
stepnumber=1,
tabsize=4,
numbersep=5pt
}
\addbibresource{../../library/bibliography.bib}

\begin{document}

\chapter{System model for driving on a straight line}
\section{Task 1}
State equations for KITT are given by:

\begin{align} \label{eq:task1-state-eq}
    m\dot{x}_2 = m\ddot{x}_1 &= -\rho \dot{x_{1}} + \tau \\
    y &= x_1
\end{align}

Where $x_n$ denote the components of the state vector $\vec{x}$, being velocity ($x_1$) and acceleration ($x_2$), $\rho$ is the friction coefficient and $\tau$ the  force generated by KITT's wheels. $y$ is the system's output. Writing these in the required form

\begin{align*}
    \dot{\vec{x}} &= \mathbf{A}\vec{x} + \mathbf{B}u \\
    \vec{y} &= \mathbf{C}\vec{x},
\end{align*}

yields:

\begin{align}\label{eq:task1-system}
    \begin{bmatrix}
        \dot{x}_1 \\ 
        \dot{x}_2
    \end{bmatrix} &= 
    \begin{bmatrix}
        0 & 1 \\
        0 & \frac{-\rho}{m}
    \end{bmatrix}
    \begin{bmatrix}
        x_1 \\ 
        x_2
    \end{bmatrix} +
    \begin{bmatrix}
        0 \\ 
        \frac{1}{m}
    \end{bmatrix}
    u
    \\
    y &= 
    \begin{bmatrix}
        1 & 0
    \end{bmatrix}
    \begin{bmatrix}
        x_1 \\ 
        x_2
    \end{bmatrix}.
\end{align}

We immediately notice this system is one of the second order, since it has two state variables. Deriving the characteristic equation $\mathrm{det}\{\mathbf{A} - \lambda\mathbf{I}\} = 0$ yields:

\begin{equation}
    \lambda(\frac{\rho}{m} + \lambda) = 0,
\end{equation}

so we obtain the eigenvalues:

\begin{equation}
    \lambda_1 = 0 \quad\quad \lambda_2 = -\frac{\rho}{m}.
\end{equation}

This renders the system either asymptotically stable, or unstable. Common sense dictates that the system is stable: KITT will not drive of by its own, but we can prove it by deriving the system's eigenvectors:

\begin{equation}
    \vec{e}_1 =
        \begin{bmatrix}
            1 \\
            0
        \end{bmatrix}
    \quad\quad
    \vec{e}_2 =
        \begin{bmatrix}
            -\frac{\rho}{m} \\
            1
        \end{bmatrix}.
\end{equation}

We can see by inspection that these vectors are independent, so the system is asymptotically stable. To determine controllability, we derive the controllability matrix:

\begin{equation}
    \mathcal{C} = 
    \begin{bmatrix}
        \mathbf{B} & \mathbf{AB}
    \end{bmatrix} =
    \begin{bmatrix}
        0 & \frac{1}{m} \\
        \frac{1}{m} & -\frac{\rho}{m} \\
    \end{bmatrix}.
\end{equation}

It is easily observed that $\mathrm{rank}(\mathcal{C}) = 2$. Combining this with the fact that $\mathbf{A}$ is a 2x2-matrix, yields that the system is controllable. For observability we can perform similar steps, deriving the observability matrix:

\begin{equation}
    \mathcal{O} = 
    \begin{bmatrix}
      \mathbf{C} \\
      \mathbf{CA}
    \end{bmatrix} = 
    \begin{bmatrix}
        1 & 0 \\
        0 & 1 \\
        0 & -\frac{1}{m}
    \end{bmatrix},
\end{equation}

for which holds that $\mathrm{rank}(\mathcal{O}) = 2$ as well, meaning the system is observable too.

\section{Task 2}
The previous equations do not give an equation for determining $\tau$, we will now derive a set of expressions, which we can use to complete the model. Using the given properties we can write the following for $\tau$:

\begin{equation}\label{eq-task2-tau}
    \tau = k_w k_g k_t i,
\end{equation}

where $k_w$ is a constant relating torque on the wheels to force, $k_g$ is the gear ratio, $k_t$ a constant that relates motor current to motor torque and $i$ the motor current. We can argue that $k_w$ must be equal to $\frac{1}{r_w}$, where $r_w$ is the wheel radius. Relating the forward velocity $\dot{x}_1$ to the velocity of the motor, we can write:

\begin{equation}\label{eq-task2-e}
    e = k_t \omega_w = \frac{\dot{x}_1}{k_g r_w},
\end{equation}

where $\omega_w$ denotes the angular velocity of KITT's wheels. Combining Equations~\ref{eq-task2-tau} and \ref{eq-task2-e} yields the following expression for $\tau$:

\begin{equation}  
    \tau = \frac{k_w k_g k_t i}{R}.
\end{equation}

In this equation we had to introduce a new state $i = x_3$, witch renders the system one of the third order. To complete the model we need to derive an expression for $\dot{x}_3 = \frac{di}{dt}$ from the voltage balance of our system:

\begin{equation}\label{eq-task2-didt}
    \frac{di}{dt} = \frac{v-iR-\frac{\dot{x}_1}{k_g r_w}}{R}.
\end{equation}

We can now write the total system in matrix-form, by setting the control input $u$ to be equal to $v$:

\begin{equation}
    \begin{bmatrix}
        \dot{x}_1 \\ 
        \dot{x}_2 \\
        \dot{x}_3
    \end{bmatrix} = 
    \begin{bmatrix}
        0 & 1 & 0 \\
        0 & \frac{-\rho}{m} & \frac{k_{t}k_{g}}{r_{w}m} \\
        0 & \frac{-k_{t}}{k_{g} r_{w} L} & -\frac{R}{L}
    \end{bmatrix}
    \begin{bmatrix}
        x_1 \\
        x_2 \\
        x_3
    \end{bmatrix} + 
    \begin{bmatrix}
        0 \\ 
        0 \\
        \frac{1}{L}
    \end{bmatrix}v.
\end{equation}

When setting $L$ to be equal to zero, we can eliminate the third state, so the system simplifies to its final form:

\begin{equation}
    \begin{bmatrix}
        \dot{x}_1 \\ 
        \dot{x}_2
    \end{bmatrix} = 
    \begin{bmatrix}
        0 & 1 \\
        0 & -\frac{\rho}{m} - \frac{k_t^2}{r_w^2m}
    \end{bmatrix}
    \begin{bmatrix}
        x_1 \\ 
        x_2
    \end{bmatrix} +
    \begin{bmatrix}
        0 \\ 
        \frac{k_g k_t}{R r_w m}
    \end{bmatrix}v.
\end{equation}

Since the form of this system is no different from the one in Equation~\ref{eq:task1-system}, we can draw the same conclusions regarding its stability, controllability and observability, being that it is stable, controllable and observable.
\end{document}